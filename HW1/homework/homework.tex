\documentclass[12pt]{article}
\usepackage{natbib}
\usepackage{url}
\usepackage{amsmath}
\usepackage{graphicx}
\graphicspath{{fig/}}
\usepackage{parskip}
\usepackage{adjustbox}
\usepackage{fancyhdr}
\usepackage{commath}%定义d
\usepackage[UTF8,heading=true]{ctex}
\usepackage{bm}
\usepackage{titlesec}
\usepackage{caption}
\usepackage{paralist}
\usepackage{multirow}
\usepackage{booktabs} % To thicken table lines
\usepackage{titletoc}
\usepackage{diagbox}
\usepackage{bm}
\usepackage{autobreak}
\usepackage{authblk}
\usepackage{indentfirst}
\usepackage{float}
\usepackage{amsthm}
\usepackage{fontspec}
\usepackage{color}
%\usepackage{txfonts} %设置字体为times new roman
\usepackage{lettrine}
\usepackage{nameref}
%\usepackage[nottoc]{tocbibind}
\usepackage{amssymb}%font
\usepackage{lipsum}%make test words
\usepackage{picinpar}%words around the picture
\usepackage[all]{xy}%draw arrow
\usepackage{asymptote}%draw picture
\usepackage[perpage]{footmisc}%脚注每页清零
\usepackage[cmyk]{xcolor}

% \geometry{bottom=2.5cm,left=2cm,right=2cm,top=2.5cm}
\newcommand{\crefrangeconjunction}{ - }
\setlength{\parindent}{2em}

\ctexset{today=big}%日期类型设置


% ======================================
% = Color de la Universidad de Sevilla =
% ======================================
\usepackage{tikz}
\definecolor{PKUred}{cmyk}{0,1,1,0.45}
%超链接设置
\usepackage[breaklinks,colorlinks,linkcolor=PKUred,citecolor=PKUred,pagebackref,urlcolor=black]{hyperref}
\usepackage{cleveref}


\renewcommand*\footnoterule{%
    \vspace*{-3pt}%
    {\color{PKUred}\hrule width 2in height 0.4pt}%
    \vspace*{2.6pt}%
}





% %%% Equation and float numbering
% \numberwithin{equation}{section}		% Equationnumbering: section.eq#
% \numberwithin{figure}{section}			% Figurenumbering: section.fig#
% \numberwithin{table}{section}				% Tablenumbering: section.tab#


%代码设置
\usepackage{listings}
\usepackage{fontspec} % 定制字体
\newfontfamily\menlo{Menlo}
\usepackage{xcolor} % 定制颜色
\definecolor{mygreen}{rgb}{0,0.6,0}
\definecolor{mygray}{rgb}{0.5,0.5,0.5}
\definecolor{mymauve}{rgb}{0.58,0,0.82}
\lstset{ %
backgroundcolor=\color{white},      % choose the background color
basicstyle=\footnotesize\ttfamily,  % size of fonts used for the code
columns=fullflexible,
tabsize=4,
breaklines=true,               % automatic line breaking only at whitespace
captionpos=b,                  % sets the caption-position to bottom
commentstyle=\color{mygreen},  % comment style
escapeinside={\%*}{*)},        % if you want to add LaTeX within your code
keywordstyle=\color{blue},     % keyword style
stringstyle=\color{mymauve}\ttfamily,  % string literal style
frame=single,
rulesepcolor=\color{red!20!green!20!blue!20},
% identifierstyle=\color{red},
language=c++,
xleftmargin=4em,xrightmargin=2em, aboveskip=1em,
framexleftmargin=2em,
numbers=left
}

%脚注
\renewcommand\thefootnote{\fnsymbol{footnote}}

%定义常数i、e、积分符号d
\newcommand\mi{\mathrm{i}}
\newcommand\me{\mathrm{e}}

%%% Maketitle metadata
\newcommand{\horrule}[1]{\rule{\linewidth}{#1}} 	% Horizontal rule
\newcommand{\tabincell}[2]{\begin{tabular}{@{}#1@{}}#2\end{tabular}}

\newcommand{\chuhao}{\fontsize{42pt}{\baselineskip}\selectfont}
\newcommand{\xiaochuhao}{\fontsize{36pt}{\baselineskip}\selectfont}
\newcommand{\yihao}{\fontsize{28pt}{\baselineskip}\selectfont}
\newcommand{\erhao}{\fontsize{21pt}{\baselineskip}\selectfont}
\newcommand{\xiaoerhao}{\fontsize{18pt}{\baselineskip}\selectfont}
\newcommand{\sanhao}{\fontsize{15.75pt}{\baselineskip}\selectfont}
\newcommand{\sihao}{\fontsize{14pt}{\baselineskip}\selectfont}
\newcommand{\xiaosihao}{\fontsize{12pt}{\baselineskip}\selectfont}
\newcommand{\wuhao}{\fontsize{10.5pt}{\baselineskip}\selectfont}
\newcommand{\xiaowuhao}{\fontsize{9pt}{\baselineskip}\selectfont}
\newcommand{\liuhao}{\fontsize{7.875pt}{\baselineskip}\selectfont}
\newcommand{\qihao}{\fontsize{5.25pt}{\baselineskip}\selectfont}
\setcounter{secnumdepth}{2}
\usepackage{bm}
\usepackage{autobreak}
\usepackage{amsmath}
\setlength{\parindent}{2em}


%pdf文件设置
\hypersetup{
	pdfauthor={袁磊祺},
	pdftitle={高等应用数学1}
}

\title{
		\vspace{-1in} 	
		\usefont{OT1}{bch}{b}{n}
		\normalfont \normalsize \textsc{\LARGE Peking University}\\[1cm] % Name of your university/college \\ [25pt]
		\horrule{0.5pt} \\[0.5cm]
		\huge \bfseries{高等应用数学1} \\
		\horrule{2pt} \\[0.5cm]
}
\author{
		\normalfont 								\normalsize
		第二组\quad 袁磊祺 \quad 刘志如 \quad 宋庭鉴 \quad 岐亦铭 \quad 董淏翔 \quad 周子铭 \quad 撒普尔\\	\normalsize
        \today
}
\date{}

\begin{document}

% %%%%%%%%%%%%%%%%%%%%%%%%%%%%%%%%%%%%%%%%%%%%%%
\renewcommand\contentsname{\vspace*{-2cm}\centerline{\textsf{目\quad 录}}\vspace*{-1.5cm}}
\renewcommand\listfigurename{插图目录}
\renewcommand\listtablename{表格目录}
\renewcommand\refname{参考文献}
\renewcommand\indexname{索引}
\renewcommand\figurename{图}
\renewcommand\tablename{表}
\renewcommand\abstractname{摘\quad 要}
\renewcommand\partname{部分}
\renewcommand\appendixname{附录}
\def\equationautorefname{式}%
\def\footnoteautorefname{脚注}%
\def\itemautorefname{项}%
\def\figureautorefname{图}%
\def\tableautorefname{表}%
\def\partautorefname{篇}%
\def\appendixautorefname{附录}%
\def\chapterautorefname{章}%
\def\sectionautorefname{节}%
\def\subsectionautorefname{小小节}%
\def\subsubsectionautorefname{subsubsection}%
\def\paragraphautorefname{段落}%
\def\subparagraphautorefname{子段落}%
\def\FancyVerbLineautorefname{行}%
\def\theoremautorefname{定理}%
\crefname{figure}{图}{图}
\crefname{equation}{式}{式}
\crefname{table}{表}{表}




\maketitle

\section{P408.13}



\subsection{(a)}

(38a)的物理意义是原来初始时刻位于$X$,并且具有相应速度$U$的微团,在时刻$t$的空间坐标.(38b)的物理意义是在时刻$t$位于$x$的具有速度$u$的微团在原来初始时刻位于$X$,所具有的速度$U$.

\subsection{(b)}

$A$的物理意义是原来初始时刻位于$(X,U)$的微团在时刻$t$的速度$u$随时间的变化率,也就是加速度.则$A$的空间形式可以定义为:
\begin{equation}
	a(x,u,t)=A[X(x,u,t), U(x,u,t)]
\end{equation}

\subsection{(c)}
物质变量表示的导数
\begin{equation}
	\frac{\partial\overline\Psi(X,U,t)}{\partial t} = \frac{\partial\Psi}{\partial x}\cdot\frac{\partial x}{\partial t} + \frac{\partial\Psi}{\partial u}\cdot\frac{\partial u}{\partial t} + \frac{\partial\Psi}{\partial t}
\end{equation}
根据$A$和$u$的定义,有
\begin{equation}
	u(X,U,t) = \frac{\partial x(X,U,t)}{\partial t}
\end{equation}
\begin{equation}
	A(X,U,t) = \frac{\partial u(X,U,t)}{\partial t}
\end{equation}
故,有
\begin{equation}
	\frac{\partial\overline\Psi(X,U,t)}{\partial t} =
\frac{\partial\Psi}{\partial t} + u\cdot\frac{\partial\Psi}{\partial x} + A\cdot\frac{\partial\Psi}{\partial u}
\end{equation}
又因为有(41)式,
\begin{equation}
	\frac{\mathscr{D}\Psi(x,u,t)}{\mathscr{D} t} =
\frac{\partial\Psi}{\partial t} + u\cdot\frac{\partial\Psi}{\partial x} + A\cdot\frac{\partial\Psi}{\partial u}\qed
\end{equation}

\subsection{(d)}

由Jacobi行列式的定义与行列式的求导公式可得:
\begin{equation}
	\frac{\partial \mathscr{J}(X,U,t)}{\partial t} =\frac{\partial}{\partial t}\cdot
{\left|
	\begin{array}{ccc}
    	\frac{\partial x}{\partial X} & \frac{\partial x}{\partial U}\\
    	\frac{\partial u}{\partial X} & \frac{\partial u}{\partial U}
	\end{array} 	
\right |} = 
{\left|
	\begin{array}{ccc}
		\frac{\partial u}{\partial X} & \frac{\partial u}{\partial U}\\
		\frac{\partial u}{\partial X} & \frac{\partial u}{\partial U}
	\end{array} 	
\right |} + 
{\left|
	\begin{array}{ccc}
		\frac{\partial x}{\partial X} & \frac{\partial x}{\partial U}\\
		\frac{\partial A}{\partial X} & \frac{\partial A}{\partial U}
	\end{array} 	
	\right |}
\end{equation}
上式右端第一个行列式为$0$,第二个行列式为
\begin{equation}
	{\left|
	\begin{array}{ccc}
		\frac{\partial x}{\partial X} & \frac{\partial x}{\partial U}\\
		\frac{\partial A}{\partial X} & \frac{\partial A}{\partial U}
	\end{array} 	
	\right |} = 
{\left|
	\begin{array}{ccc}
		\frac{\partial x}{\partial X} & \frac{\partial x}{\partial U}\\
		\frac{\partial a}{\partial u}\cdot\frac{\partial u}{\partial X} & \frac{\partial a}{\partial u}\cdot\frac{\partial u}{\partial U}
	\end{array} 	
	\right |} = 
\frac{\partial a}{\partial u}\cdot \mathscr{J}
\end{equation}
\begin{equation}
{\left|
	\begin{array}{ccc}
		\frac{\partial x}{\partial X} & \frac{\partial x}{\partial U}\\
		\frac{\partial u}{\partial X} & \frac{\partial u}{\partial U}
	\end{array} 	
	\right |} = 
\mathscr{J}(X,U,t)
\end{equation}
所以,$\frac{\partial \mathscr{J}}{\partial t} = \mathscr{J}\cdot\frac{\partial a}{\partial u}$.\qed

\subsection{(e)}

为了将对时间的导数放到积分号里,将空间描述换为物质描述,使得积分区域与时间无关.
\begin{equation}
	\begin{split}
		\frac{\dif }{\dif t}\iint_{R(t)} \Psi(x,u,t)\dif x\dif u = & \frac{\dif }{\dif t} \iint_{R(0)} \overline\Psi(X,U,t)\mathscr{J}(X,U,t)\dif X\dif U\\
		= &  \iint_{R(0)} \left[ \pd{\overline\Psi(X,U,t)}{t} \mathscr{J} + \overline\Psi(X,U,t)\pd{\mathscr{J}}{t} \right] \dif X\dif U\\
		= & \iint_{R(0)} \left[ \pd{\overline\Psi(X,U,t)}{t} \mathscr{J} + \overline\Psi(X,U,t)\mathscr{J}\frac{\partial a}{\partial u} \right] \dif X\dif U\\
		= & \iint_{R(t)} \left[ \frac{\mathscr{D} \Psi(X,U,t)}{\mathscr{D} t}  + \Psi(X,U,t)\frac{\partial a}{\partial u} \right]\dif x\dif u
	\end{split}
\end{equation}
所以
\begin{equation}
	\frac{\mathscr{D} \Psi(X,U,t)}{\mathscr{D} t}  + \Psi(X,U,t)\frac{\partial a}{\partial u} = 0
\end{equation}

\section{P416.4}

\subsection{(a)}

有质量守恒方程:
\begin{equation}
	\frac{\partial \rho}{\partial t} + \frac{\partial(\rho v_i)}{\partial x_i}=0
\end{equation}
物质导数的分量形式为:
\begin{equation}
	\rho\frac{\Dif  v_j}{\Dif t}=\rho\frac{\partial v_j}{\partial t} + \rho v_i\frac{\partial v_j}{\partial x_i}
\end{equation}
将质量守恒方程代入以上方程,可得
\begin{equation}
	\rho\frac{\Dif  v_j}{\Dif t}=\rho\frac{\partial v_j}{\partial t} + \rho v_i\frac{\partial v_j}{\partial x_i} + v_j \left[\frac{\partial \rho}{\partial t} + \frac{\partial(\rho v_i)}{\partial x_i}\right]
\end{equation}
整理上式,即可得出方程
\begin{equation}
	\rho\frac{\Dif  v_j}{\Dif t}=\frac{\partial(\rho v_j)}{\partial t}+\frac{\partial (v_j\rho v_i)}{\partial x_i}\qed
\end{equation}
首先将(19)式写为积分形式,可得
\begin{equation}
	\iiint_{R}\Big[\rho\frac{\Dif  v_j}{\Dif t} - \rho f_j - \frac{\partial(T_{ij})}{\partial x_i}\Big] \dif \tau = 0
\end{equation}
将上式中的物质导数用之前推导出的方程代替,并用方向符号写成
\begin{equation}
\iiint_{R} \frac{\partial (\rho \bm{v})}{\partial t} \dif \tau = \iiint_{R} \Big[\rho f + \nabla\cdot \bm{T} - \nabla\cdot(\rho \bm{v}\bm{v})\Big]\dif \tau
\end{equation}
因为区域$R$固定,上式左端对于时间的倒数可以提取到积分号外,并利用奥高公式,将$ \nabla\cdot$换为$ n\cdot$,则
\begin{equation}
	\od{}{t}\iiint_{R} \rho \bm{v} \dif \tau = \iiint_{R} \rho \bm{f} \dif \tau + \oiint_{\partial R} \left[ \bm{t} - \rho \bm{v}(\bm{v}\cdot \bm{n}) \right] \dif \sigma  \qed
\end{equation}

\subsection{(b)}

不难理解, $\rho v_iv_j$是通过控制面单位面积的动量通量。例如, $\rho u_1$是通过法向向量为$\bm{e}_1$面的质量通量,这部分质量通量携带的动量为$\rho u_1 \bm{u} \dif x_2 \dif x_3$.

动量流量张量是由固定区域边界上的流入或流出得物质所携带的动量产生的.对该固定区域的作用与作用在该固定边界上的应力相似,动量流量本身也具有不同方向的分量,而动量流量张量的分量即表示动量流量本身的分量在指定坐标系下不同方向的分量.

\section{P434.5}

\subsection{(a)}

质量守恒方程、线动量平衡定律和能量平衡定律分别是以下形式:
\begin{equation}
	\frac{\Dif \rho}{\Dif t} = -\rho\nabla\cdot \bm{v}
\end{equation}
\begin{equation}
	\rho\frac{\Dif \bm{v}}{\Dif t} = \rho \bm{f} + \nabla\cdot \bm{T}
\end{equation}
\begin{equation}
	\rho\frac{\Dif }{\Dif t}\left(\frac{1}{2}\bm{v}\cdot \bm{v}+e\right) = \rho(\bm{f}\cdot \bm{v})-\nabla\cdot(\bm{q}-\bm{T}\cdot \bm{v})
\end{equation}
在物质区域之中物理量的改变率来自体力和面力的贡献.对于质量守恒方程而言,能使得质量增加或者减少的体力是不存在的,质量的改变只能由面通量改变,所以就自然没有Q项.

\subsection{(b)}

微分形式的方程推导过程中,对积分方程的边界场使用了高斯公式,然后用D-R引理将积分方程变为了微分方程
\begin{equation}
	\oiint_{\partial R}\bm{n}\cdot \bm{J} \dif \sigma = \iiint_{R}\nabla\cdot \bm{J}\dif \tau
\end{equation}
故由上述方程的左边可以看出J的物理意义是$\partial R$界面上的通量.在质量守恒方程中, $\bm{J}=\bm{v}$, $\bm{v}$可解释为质量通量矢量,能量平衡方程中, $\bm{J} = \bm{q}-\bm{T}\cdot \bm{v}$则可以解释为能量通量矢量.

\subsection{(c)}

在(40b)式中,
\begin{equation}
	\frac{\dif }{\dif t}\iiint_{R(t)}\rho F\dif \tau = 
\iiint_{R(t)}\rho Q\dif \tau + \iint_{\partial R(t)}j(\bm{n}) \dif \sigma
\end{equation}

$F$表示单位质量物理具有的物理量$F$(例如动量能量),等式左侧表示$t$时刻占据区域$R(t)$具有的物理量随时间的变化率.等式右侧第一项表示区域$R(t)$内的物理量$F$的生成速率, $Q$是单位质量物理量$F$的生成速率.等式右侧第二项表示边界上的物理量$F$的流入速率, $\bm{n}$是内法线方向,流入的量等于从反方向看流出的量,所以有$j(-\bm{n})=-j(\bm{n})$. $j(\bm{n})\equiv j(\bm{x},t,\bm{n})$是在$\partial R$上$t$时刻,位于$\bm{x}$处沿$\bm{n}$方向物理量$F$的通量.

\nocite{*}

\bibliographystyle{plain}
\phantomsection

\addcontentsline{toc}{section}{参考文献} %向目录中添加条目,以章的名义
\bibliography{homework}

\end{document}
