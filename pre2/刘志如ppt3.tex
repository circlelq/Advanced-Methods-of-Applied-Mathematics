\documentclass{beamer}
\usepackage{CJK}
\usefonttheme[onlymath]{serif}
\usepackage[greek,english]{babel}
\usepackage{bm}

\begin{document}
\begin{CJK}{UTF8}{song}

\begin{frame}[allowframebreaks]
\frametitle{傅里叶级数简单示例}

方形波函数
\begin{equation}
S(x)=\frac{4}{\pi}\left(\sin x+\frac{1}{3} \sin 3 x+\frac{1}{5} \sin 5 x+\cdots\right)
\end{equation}

绝对值函数
\begin{equation}
C(x)=|x|=\pi-\frac{4}{\pi}\left(\frac{\cos x}{1^{2}}+\frac{\cos 3 x}{3^{2}}+\frac{\cos 5 x}{5^{2}}+\cdots\right)
\end{equation}

\begin{equation}
C^{\prime}(x)=S(x)
\end{equation}
\end{frame}

\begin{frame}[allowframebreaks]
\frametitle{傅里叶级数的积分和微分}

分段光滑的函数可用下面的傅里叶级数表示
\begin{equation}
f(x)=\frac{a_{0}}{2}+\sum_{n=1}^{\infty}\left(a_{n} \cos n x+b_{n} \sin n x\right)
\end{equation}

形式积分
\begin{equation}
\int_{0}^{x} f(t) d t-\frac{a_{0}}{2} x \sim \sum_{n=1}^{\infty}\left[\frac{a_{n}}{n} \sin n x+\frac{b_{n}}{n}(1-\cos n x)\right]
\end{equation}

可用下式验证
\begin{equation}
\int_{0}^{x} f(t) d t-\frac{a_{0}}{2} x=\frac{A_{0}}{2}+\sum_{n=1}^{\infty}\left(A_{n} \cos n x+B_{n} \sin n x\right)
\end{equation}

形式微分
\begin{equation}
f^{\prime}(x) \sim 0+\sum_{n=1}^{\infty}\left(-n a_{n} \sin n x+n b_{n} \cos n x\right)
\end{equation}

若用下式验证
\begin{equation}
\frac{a_{0}^{\prime}}{2}+\sum_{n=1}^{\infty}\left(a_{n}^{\prime} \cos n x+b_{n}^{\prime} \sin n x\right)
\end{equation}

得约束条件之一
\begin{equation}
f(\pi)=f(-\pi)
\end{equation}
\end{frame}

\begin{frame}[allowframebreaks]
\frametitle{吉布斯现象}

部分和在间断点附近出现称作吉布斯现象的异常行为

以方波函数为例,可证明,仅当N趋于无穷时,部分和才会趋近于极限值

\begin{equation}
S_{N}(x) \approx \frac{2}{\pi} \int_{0}^{m x} \frac{\sin \eta}{\eta} \frac{\eta / 2 m}{\sin (\eta / 2 m)} d \eta
\end{equation}

其中
\begin{equation}
m=N+\frac{1}{2}
\end{equation}

\begin{equation}
S_{\max }=\frac{2}{\pi}\left(\int_{0}^{\left(N+\frac{1}{2}\right) x} \frac{\sin \eta}{\eta} d \eta\right)_{\max } \approx 1.179
\end{equation}

(这里麻烦circle抠一下128页中间的那三张配图,仅图片即可,谢谢)
\end{frame}

\begin{frame}[allowframebreaks]
\frametitle{具有最小二乘误差的近似}

均方误差取极小值
\begin{equation}
M=\frac{1}{2 \pi} \int_{-\pi}^{\pi} \varepsilon_{N}^{2} d x
\end{equation}

\begin{equation}
\varepsilon_{N}=f(x)-\sum_{n=-N}^{N} \gamma_{n} e^{i n x}
\end{equation}

必要条件
\begin{equation}
\gamma_{n}=\frac{1}{2 \pi} \int_{-\pi}^{\pi} f(x) e^{-i n x} d x
\end{equation}

充分条件
\begin{equation}
c_{n}=\gamma_{n}
\end{equation}
\end{frame}

\begin{frame}[allowframebreaks]
\frametitle{具有最小二乘误差的近似}

最小二乘法可推广

正交条件
\begin{equation}
\int_{-\pi}^{\pi} \phi_{m}(x) \phi_{n}(x) w(x) d x=\delta_{m n}
\end{equation}

其中w为非负的权函数

加权均方误差取最小值
\begin{equation}
\varepsilon_{N}=f(x)-\sum_{n=0}^{N} \gamma_{n} \phi_{n}(x)
\end{equation}

\begin{equation}
M_{w}=\frac{\int_{-\pi}^{\pi} \varepsilon_{N}^{2}(x) w(x) d x}{\int_{-\pi}^{\pi} w(x) d x}
\end{equation}
\end{frame}

\begin{frame}[allowframebreaks]
\frametitle{贝塞尔不等式和Parseval定理}

\begin{equation}
M=\left\langle f^{2}\right\rangle-\sum_{n=-N}^{N}\left|c_{n}\right|^{2}+\sum_{n=-N}^{N}\left|c_{n}-\gamma_{n}\right|^{2}
\end{equation}

由于条件$c_{n}=\gamma_{n}$和$M \geq 0$

\begin{equation}
\sum_{n=-N}^{N}\left|c_{n}\right|^{2} \leq\left\langle f^{2}\right\rangle
\end{equation}

假如除了有限多个间断点意外,傅里叶级数按点态方式收敛,则
\begin{equation}
\lim _{N \rightarrow \infty} M=0
\end{equation}

\begin{equation}
\sum_{n=-\infty}^{\infty}\left|c_{n}\right|^{2}=\left\langle f^{2}\right\rangle
\end{equation}

Riesz-Fischer定理是Parseval定理的逆问题
\begin{equation}
\frac{1}{2} a_{0}^{2}+\sum_{m=1}^{\infty}\left(a_{m}^{2}+b_{m}^{2}\right)
\end{equation}

\begin{equation}
\frac{a_{0}}{2}+\sum_{m=1}^{\infty}\left(a_{m} \cos m x+b_{m} \sin m x\right)
\end{equation}

\end{frame}

\begin{frame}[allowframebreaks]
\frametitle{Parseval定理的应用举例}

根据普朗克辐射定律计算斯蒂芬辐射常数时,需要计算积分
\begin{equation}
I=\int_{0}^{\infty} \frac{x^{3}}{e^{x}-1} d x
\end{equation}

\begin{equation}
1+\frac{1}{3^{4}}+\frac{1}{5^{4}}+\cdots=\frac{\pi^{4}}{96}
\end{equation}

\begin{equation}
I=\Gamma(4)\left(1+\frac{1}{2^{4}}+\frac{1}{3^{4}}+\cdots\right)
\end{equation}

\begin{equation}
I=\frac{\frac{\pi^{4}}{96}}{1-\frac{1}{2^{4}}} \times 6=\frac{\pi^{4}}{15}
\end{equation}
\end{frame}

\end{CJK}
\end{document}}