\documentclass[12pt]{article}
\usepackage{natbib}
\usepackage{url}
\usepackage{amsmath}
\usepackage{graphicx}
\graphicspath{{fig/}}
\usepackage{parskip}
\usepackage{adjustbox}
\usepackage{fancyhdr}
\usepackage{commath}%定义d
\usepackage[UTF8,heading=true]{ctex}
\usepackage{bm}
\usepackage{titlesec}
\usepackage{caption}
\usepackage{paralist}
\usepackage{multirow}
\usepackage{booktabs} % To thicken table lines
\usepackage{titletoc}
\usepackage{diagbox}
\usepackage{bm}
\usepackage{autobreak}
\usepackage{authblk}
\usepackage{indentfirst}
\usepackage{float}
\usepackage{amsthm}
\usepackage{fontspec}
\usepackage{color}
%\usepackage{txfonts} %设置字体为times new roman
\usepackage{lettrine}
\usepackage{nameref}
%\usepackage[nottoc]{tocbibind}
\usepackage{amssymb}%font
\usepackage{lipsum}%make test words
\usepackage{picinpar}%words around the picture
\usepackage[all]{xy}%draw arrow
\usepackage{asymptote}%draw picture
\usepackage[perpage]{footmisc}%脚注每页清零
\usepackage[cmyk]{xcolor}

% \geometry{bottom=2.5cm,left=2cm,right=2cm,top=2.5cm}
\newcommand{\crefrangeconjunction}{ - }
\setlength{\parindent}{2em}

\ctexset{today=big}%日期类型设置


% ======================================
% = Color de la Universidad de Sevilla =
% ======================================
\usepackage{tikz}
\definecolor{PKUred}{cmyk}{0,1,1,0.45}
%超链接设置
\usepackage[breaklinks,colorlinks,linkcolor=PKUred,citecolor=PKUred,pagebackref,urlcolor=black]{hyperref}
\usepackage{cleveref}


\renewcommand*\footnoterule{%
    \vspace*{-3pt}%
    {\color{PKUred}\hrule width 2in height 0.4pt}%
    \vspace*{2.6pt}%
}





% %%% Equation and float numbering
% \numberwithin{equation}{section}		% Equationnumbering: section.eq#
% \numberwithin{figure}{section}			% Figurenumbering: section.fig#
% \numberwithin{table}{section}				% Tablenumbering: section.tab#


%代码设置
\usepackage{listings}
\usepackage{fontspec} % 定制字体
\newfontfamily\menlo{Menlo}
\usepackage{xcolor} % 定制颜色
\definecolor{mygreen}{rgb}{0,0.6,0}
\definecolor{mygray}{rgb}{0.5,0.5,0.5}
\definecolor{mymauve}{rgb}{0.58,0,0.82}
\lstset{ %
backgroundcolor=\color{white},      % choose the background color
basicstyle=\footnotesize\ttfamily,  % size of fonts used for the code
columns=fullflexible,
tabsize=4,
breaklines=true,               % automatic line breaking only at whitespace
captionpos=b,                  % sets the caption-position to bottom
commentstyle=\color{mygreen},  % comment style
escapeinside={\%*}{*)},        % if you want to add LaTeX within your code
keywordstyle=\color{blue},     % keyword style
stringstyle=\color{mymauve}\ttfamily,  % string literal style
frame=single,
rulesepcolor=\color{red!20!green!20!blue!20},
% identifierstyle=\color{red},
language=c++,
xleftmargin=4em,xrightmargin=2em, aboveskip=1em,
framexleftmargin=2em,
numbers=left
}

%脚注
\renewcommand\thefootnote{\fnsymbol{footnote}}

%定义常数i、e、积分符号d
\newcommand\mi{\mathrm{i}}
\newcommand\me{\mathrm{e}}

%%% Maketitle metadata
\newcommand{\horrule}[1]{\rule{\linewidth}{#1}} 	% Horizontal rule
\newcommand{\tabincell}[2]{\begin{tabular}{@{}#1@{}}#2\end{tabular}}

\newcommand{\chuhao}{\fontsize{42pt}{\baselineskip}\selectfont}
\newcommand{\xiaochuhao}{\fontsize{36pt}{\baselineskip}\selectfont}
\newcommand{\yihao}{\fontsize{28pt}{\baselineskip}\selectfont}
\newcommand{\erhao}{\fontsize{21pt}{\baselineskip}\selectfont}
\newcommand{\xiaoerhao}{\fontsize{18pt}{\baselineskip}\selectfont}
\newcommand{\sanhao}{\fontsize{15.75pt}{\baselineskip}\selectfont}
\newcommand{\sihao}{\fontsize{14pt}{\baselineskip}\selectfont}
\newcommand{\xiaosihao}{\fontsize{12pt}{\baselineskip}\selectfont}
\newcommand{\wuhao}{\fontsize{10.5pt}{\baselineskip}\selectfont}
\newcommand{\xiaowuhao}{\fontsize{9pt}{\baselineskip}\selectfont}
\newcommand{\liuhao}{\fontsize{7.875pt}{\baselineskip}\selectfont}
\newcommand{\qihao}{\fontsize{5.25pt}{\baselineskip}\selectfont}
\setcounter{secnumdepth}{2}
\usepackage{bm}
\usepackage{autobreak}
\usepackage{amsmath}
\setlength{\parindent}{2em}


%pdf文件设置
\hypersetup{
	pdfauthor={袁磊祺},
	pdftitle={高等应用数学作业5}
}

\title{
		\vspace{-1in} 	
		\usefont{OT1}{bch}{b}{n}
		\normalfont \normalsize \textsc{\LARGE Peking University}\\[1cm] % Name of your university/college \\ [25pt]
		\horrule{0.5pt} \\[0.5cm]
		\huge \bfseries{高等应用数学作业5} \\
		\horrule{2pt} \\[0.5cm]
}
\author{
		\normalfont 								\normalsize
		第二组\quad 袁磊祺 \quad 刘志如 \quad 宋庭鉴 \quad 岐亦铭 \quad 董淏翔 \quad 周子铭 \quad 撒普尔\\	\normalsize
        \today
}
\date{}

\begin{document}

\input{setc.tex}

\maketitle

\section{1}
Ch. 11 Sec. 11.3 Prom. 3

$\ddot{y} + \epsilon\dot{y} + y = 0,\ y(0)=0,\ \dot{y}(0)=1$

忽略小量$\epsilon$,则$y=\sin(t)$。

故$f^{(0)}(t,\tau) = A(\tau) \sin(t)$ 且 $A(0)=1$

$y(t,\tau) = f^{(0)}(t,\tau) + \epsilon f^{(1)}(t,\tau) + O(\epsilon^2)$, $\tau = \epsilon t$

$\dot{y}(t,\tau) = f^{(0)}_1(t,\tau) + \epsilon ( f^{(1)}_1(t,\tau) + f^{(0)}_2(t,\tau) ) +O(\epsilon^2)$

$\ddot{y}(t,\tau) = f^{(0)}_{11}(t,\tau) + \epsilon ( f^{(1)}_{11}(t,\tau) + 2f^{(0)}_{12}(t,\tau) ) + O(\epsilon^2)$

$\ddot{y} + \epsilon\dot{y} + y = f^{(0)}_{11}(t,\tau) + f^{(0)}(t,\tau) + \epsilon ( f^{(1)}_{11}(t,\tau) + 2f^{(0)}_{12}(t,\tau) + f^{(0)}_1(t,\tau) + f^{(1)}(t,\tau) )  + O(\epsilon^2) = 0 $

初始边界条件$f^{(0)}(0,0)=f^{(1)}(0,0)=\cdots=0$,$f^{(0)}_1(0,0)=1$,$f^{(1)}_1(0,0)+f^{(0)}_2(0,0)=0$, $\cdots$

方程展开式条件:

\begin{equation}
	\left\{
	\begin{array}{lll}
	0&=f^{(0)}_{11}(t,\tau) + f^{(0)}(t,\tau)=-A(\tau) \sin(t) + A(\tau) \sin(t)\\
	0&=f^{(1)}_{11}(t,\tau) + 2f^{(0)}_{12}(t,\tau) + f^{(0)}_1(t,\tau) + f^{(1)}(t,\tau)\\
	&=f^{(1)}_{11}(t,\tau) + f^{(1)}(t,\tau) + A(\tau)^3 \cos^3(t) + 2A^\prime(\tau) \cos(t)
	\end{array}
	\right.
\end{equation}


消除共振项$A(\tau)^3 \cos^3(t) + 2A^\prime(\tau) \cos(t) = 0 =(3A^3/4+2A^\prime)\cos(t) + A^3/4 \cos(3t)$

$3A^3/4+2A^\prime=0$ 且 $A(0)=1$。

得$A(\tau)=(1+3\tau/4)^{-0.5}$,$y \approx (1+3\epsilon t/4)^{-0.5} \sin(t) + O(\epsilon)$

解释:$\dot{y}$项系数小,针对快变量小尺度时间忽略该项,得到解主要受$\ddot{y}$与$y$项影响的周期函数,在大尺度时间下慢变量的影响反映在解的幅值随时间逐渐下降。


\section{3}
Ch. 11 Sec. 11.3 Prob.7 

\subsection{}

\begin{equation}
	\varepsilon \od[2]{y}{x} +2 \od{y}{x} +y=0 ;\quad y(0) = 0,\quad y(1) = 1, \quad 0<x<1, \quad 0<\varepsilon \ll 1
	\label{eq:3.1}
\end{equation}
假设
\begin{equation}
	y = f(x,X,\varepsilon) = f^{(0)}(x,X) + \varepsilon f^{(1)}(x,X) + \cdots
	\label{eq:3.2}
\end{equation}
其中$X=x/\varepsilon$,那么
\begin{equation}
	\od{}{x} f(x,X,\varepsilon) = \frac{1}{\varepsilon} f^{(0)}_2 + f^{(0)}_1 + f^{(1)}_2 + \varepsilon f^{(1)}_1
	\label{eq:3.3}
\end{equation}
\begin{equation}
	\od[2]{}{x} f(x,X,\varepsilon) = \frac{1}{\varepsilon^2} f^{(0)}_{22} + \frac{1}{\varepsilon} \left[ 2f^{(0)}_{21} + f^{(1)}_{22} \right] + f^{(0)}_{11} + 2 f^{(1)}_{12} + \varepsilon f^{(1)}_{12}
	\label{eq:3.4}
\end{equation}
把\cref{eq:3.1,eq:3.2,eq:3.3}代入\cref{eq:3.1}可得
\begin{equation}
	\frac{1}{\varepsilon} f^{(0)}_{22} + 2f^{(0)}_{21} + f^{(1)}_{22}  + \varepsilon \left[ f^{(0)}_{11} + 2 f^{(1)}_{12} \right] + \varepsilon^2 f^{(1)}_{12} + \frac{2}{\varepsilon} f^{(0)}_2 + 2f^{(0)}_1 + 2f^{(1)}_2 + 2\varepsilon f^{(1)}_1 + f^{(0)} + \varepsilon f^{(1)}= 0
\end{equation}
$O(\varepsilon)$项为$0$,所以
\begin{equation}
	f^{(0)}_{22} + 2f^{(0)}_2 = 0
	\label{eq:3.5}
\end{equation}
求解\cref{eq:3.5}可得
\begin{equation}
	f^{(0)}(x,X) = C_1(x) + C_2(x) \exp{(-2X)}.
	\label{eq:3.6}
\end{equation}

\subsection{}
$O(1)$项为$0$,所以
\begin{equation}
	2f^{(0)}_{21} + f^{(1)}_{22} + 2f^{(0)}_1 + 2f^{(1)}_2 = 0
	\label{eq:3.7}
\end{equation}
\cref{eq:3.6}代入\cref{eq:3.7}可得
\begin{equation}
	-2C_2'\exp{(-2X)} + f^{(1)}_{22} + 2C_1' + 2f^{(1)}_2 + C_1 + C_2 \exp{(-2X)}= 0
\end{equation}
即
\begin{equation}
	f^{(1)}_{22} + 2f^{(1)}_2 = (2C_2' - C_2)\exp{(-2X)} - (2C_1' + C_1)
\end{equation}
因为对于任意的$\varepsilon$,都要有$O(1)=0$,所以有
\begin{equation}
	2C_1' + C_1 = 0
	\label{eq:3.8}
\end{equation}
因为$\varepsilon$非常地小,$(2C_2' - C_2)\exp{(-2X)}$也很小,所以并不要求$(2C_2' - C_2)\exp{(-2X)} = 0$。由\cref{eq:3.8}可以解得
\begin{equation}
	C_1 = \exp{\left[A - \frac{1}{2}x\right]}
	\label{eq:3.9}
\end{equation}
由初始条件可得
\begin{equation}
	C_1(1) = 1
\end{equation}
所以
\begin{equation}
	C_1 = \exp{\left[\frac{1}{2}(1 - x)\right]}
\end{equation}
又根据初始条件$y(0) = 0$
\begin{equation}
	C_1(0) + C_2(0) = 0
\end{equation}
所以
\begin{equation}
	\me{^\frac{1}{2}} + C_2(0) = 0
\end{equation}
\begin{equation}
	C_2(0) = - \me{^\frac{1}{2}} 
\end{equation}

\cref{eq:3.9} 和9.2.5 远离边界层$x=0$的解是一样的。 

\subsection{}
$\varepsilon < 0$时,$(2C_2' - C_2)$必须为0,最终解得
\begin{equation}
	y(x,\varepsilon) = \me^{(2/\varepsilon)(1-x)}
\end{equation}






\nocite{*}

\input{bib.tex}

\end{document}
