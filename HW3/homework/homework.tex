\documentclass[12pt]{article}
\usepackage{natbib}
\usepackage{url}
\usepackage{amsmath}
\usepackage{graphicx}
\graphicspath{{fig/}}
\usepackage{parskip}
\usepackage{adjustbox}
\usepackage{fancyhdr}
\usepackage{commath}%定义d
\usepackage[UTF8,heading=true]{ctex}
\usepackage{bm}
\usepackage{titlesec}
\usepackage{caption}
\usepackage{paralist}
\usepackage{multirow}
\usepackage{booktabs} % To thicken table lines
\usepackage{titletoc}
\usepackage{diagbox}
\usepackage{bm}
\usepackage{autobreak}
\usepackage{authblk}
\usepackage{indentfirst}
\usepackage{float}
\usepackage{amsthm}
\usepackage{fontspec}
\usepackage{color}
%\usepackage{txfonts} %设置字体为times new roman
\usepackage{lettrine}
\usepackage{nameref}
%\usepackage[nottoc]{tocbibind}
\usepackage{amssymb}%font
\usepackage{lipsum}%make test words
\usepackage{picinpar}%words around the picture
\usepackage[all]{xy}%draw arrow
\usepackage{asymptote}%draw picture
\usepackage[perpage]{footmisc}%脚注每页清零
\usepackage[cmyk]{xcolor}

% \geometry{bottom=2.5cm,left=2cm,right=2cm,top=2.5cm}
\newcommand{\crefrangeconjunction}{ - }
\setlength{\parindent}{2em}

\ctexset{today=big}%日期类型设置


% ======================================
% = Color de la Universidad de Sevilla =
% ======================================
\usepackage{tikz}
\definecolor{PKUred}{cmyk}{0,1,1,0.45}
%超链接设置
\usepackage[breaklinks,colorlinks,linkcolor=PKUred,citecolor=PKUred,pagebackref,urlcolor=black]{hyperref}
\usepackage{cleveref}


\renewcommand*\footnoterule{%
    \vspace*{-3pt}%
    {\color{PKUred}\hrule width 2in height 0.4pt}%
    \vspace*{2.6pt}%
}





% %%% Equation and float numbering
% \numberwithin{equation}{section}		% Equationnumbering: section.eq#
% \numberwithin{figure}{section}			% Figurenumbering: section.fig#
% \numberwithin{table}{section}				% Tablenumbering: section.tab#


%代码设置
\usepackage{listings}
\usepackage{fontspec} % 定制字体
\newfontfamily\menlo{Menlo}
\usepackage{xcolor} % 定制颜色
\definecolor{mygreen}{rgb}{0,0.6,0}
\definecolor{mygray}{rgb}{0.5,0.5,0.5}
\definecolor{mymauve}{rgb}{0.58,0,0.82}
\lstset{ %
backgroundcolor=\color{white},      % choose the background color
basicstyle=\footnotesize\ttfamily,  % size of fonts used for the code
columns=fullflexible,
tabsize=4,
breaklines=true,               % automatic line breaking only at whitespace
captionpos=b,                  % sets the caption-position to bottom
commentstyle=\color{mygreen},  % comment style
escapeinside={\%*}{*)},        % if you want to add LaTeX within your code
keywordstyle=\color{blue},     % keyword style
stringstyle=\color{mymauve}\ttfamily,  % string literal style
frame=single,
rulesepcolor=\color{red!20!green!20!blue!20},
% identifierstyle=\color{red},
language=c++,
xleftmargin=4em,xrightmargin=2em, aboveskip=1em,
framexleftmargin=2em,
numbers=left
}

%脚注
\renewcommand\thefootnote{\fnsymbol{footnote}}

%定义常数i、e、积分符号d
\newcommand\mi{\mathrm{i}}
\newcommand\me{\mathrm{e}}

%%% Maketitle metadata
\newcommand{\horrule}[1]{\rule{\linewidth}{#1}} 	% Horizontal rule
\newcommand{\tabincell}[2]{\begin{tabular}{@{}#1@{}}#2\end{tabular}}

\newcommand{\chuhao}{\fontsize{42pt}{\baselineskip}\selectfont}
\newcommand{\xiaochuhao}{\fontsize{36pt}{\baselineskip}\selectfont}
\newcommand{\yihao}{\fontsize{28pt}{\baselineskip}\selectfont}
\newcommand{\erhao}{\fontsize{21pt}{\baselineskip}\selectfont}
\newcommand{\xiaoerhao}{\fontsize{18pt}{\baselineskip}\selectfont}
\newcommand{\sanhao}{\fontsize{15.75pt}{\baselineskip}\selectfont}
\newcommand{\sihao}{\fontsize{14pt}{\baselineskip}\selectfont}
\newcommand{\xiaosihao}{\fontsize{12pt}{\baselineskip}\selectfont}
\newcommand{\wuhao}{\fontsize{10.5pt}{\baselineskip}\selectfont}
\newcommand{\xiaowuhao}{\fontsize{9pt}{\baselineskip}\selectfont}
\newcommand{\liuhao}{\fontsize{7.875pt}{\baselineskip}\selectfont}
\newcommand{\qihao}{\fontsize{5.25pt}{\baselineskip}\selectfont}
\setcounter{secnumdepth}{2}
\usepackage{bm}
\usepackage{autobreak}
\usepackage{amsmath}
\setlength{\parindent}{2em}


%pdf文件设置
\hypersetup{
	pdfauthor={袁磊祺},
	pdftitle={高等应用数学作业3}
}

\title{
		\vspace{-1in} 	
		\usefont{OT1}{bch}{b}{n}
		\normalfont \normalsize \textsc{\LARGE Peking University}\\[1cm] % Name of your university/college \\ [25pt]
		\horrule{0.5pt} \\[0.5cm]
		\huge \bfseries{高等应用数学作业3} \\
		\horrule{2pt} \\[0.5cm]
}
\author{
		\normalfont 								\normalsize
		第二组\quad 袁磊祺 \quad 刘志如 \quad 宋庭鉴 \quad 岐亦铭 \quad 董淏翔 \quad 周子铭 \quad 撒普尔\\	\normalsize
        \today
}
\date{}

\begin{document}

%%%%%%%%%%%%%%%%%%%%%%%%%%%%%%%%%%%%%%%%%%%%%%
\renewcommand\contentsname{\vspace*{-2cm}\centerline{\textsf{目\quad 录}}\vspace*{-1.5cm}}
\renewcommand\listfigurename{插图目录}
\renewcommand\listtablename{表格目录}
\renewcommand\refname{参考文献}
\renewcommand\indexname{索引}
\renewcommand\figurename{图}
\renewcommand\tablename{表}
\renewcommand\abstractname{摘\quad 要}
\renewcommand\partname{部分}
\renewcommand\appendixname{附录}
\def\equationautorefname{式}%
\def\footnoteautorefname{脚注}%
\def\itemautorefname{项}%
\def\figureautorefname{图}%
\def\tableautorefname{表}%
\def\partautorefname{篇}%
\def\appendixautorefname{附录}%
\def\chapterautorefname{章}%
\def\sectionautorefname{节}%
\def\subsectionautorefname{小小节}%
\def\subsubsectionautorefname{subsubsection}%
\def\paragraphautorefname{段落}%
\def\subparagraphautorefname{子段落}%
\def\FancyVerbLineautorefname{行}%
\def\theoremautorefname{定理}%
\crefname{figure}{图}{图}
\crefname{equation}{式}{式}
\crefname{table}{表}{表}




\maketitle

\section{1}

sec. 9.2 10

外部
\begin{equation}
	1 - (y'_0)^2,
\end{equation}
\begin{equation}
	y'_0 = \pm 1.
\end{equation}
因为
\begin{equation}
	\lim_{x\to \infty} y'(x) = 1,
\end{equation}
所以
\begin{gather}
	y'_0 = 1,\\
	y_0 = x + C.
\end{gather}
第一项为$x$, 令$\xi = \frac{x}{\delta}$, 则
\begin{equation}
	\frac{\varepsilon}{\delta^3} \od[3]{y}{\xi} + \frac{\varepsilon}{\delta^2} y \od[2]{y}{\xi} + 1 - \frac{1}{\delta^2} \left(\od{y}{\xi}\right)^2 = 0
\end{equation}

\begin{enumerate}
	\item $\frac{\varepsilon}{\delta^3} = \frac{\varepsilon}{\delta^2} \Rightarrow \delta = \varepsilon$, $\frac{1}{\delta^2}$ 不是小量,不符.
	\item $\frac{\varepsilon}{\delta^3} = 1 \Rightarrow \delta = \varepsilon^{\frac{1}{3}}$, $\frac{\varepsilon}{\delta^2}$ 不是小量,不符.
	\item $\frac{\varepsilon}{\delta^3} = \frac{1}{\delta^2} \Rightarrow \delta = \varepsilon$, $\frac{1}{\delta^2}$ 则有$\frac{\varepsilon}{\delta^3} \od[3]{y}{\xi} - \frac{1}{\delta^2} \left(\od{y}{\xi}\right)^2 = 0$
	\item $\frac{\varepsilon}{\delta^2} = 1 \Rightarrow \delta = \varepsilon^{\frac{1}{2}}$, $\frac{\varepsilon}{\delta^3}$ 不是小量,不符.
	\item $\frac{\varepsilon}{\delta^2} = \frac{1}{\delta^2} \Rightarrow \varepsilon = 1$, $\frac{1}{\delta^3}$ 不是小量,不符.
\end{enumerate}
边界条件:
\begin{equation}
	y_1(0) = y'_1(0) = 0.
\end{equation}
匹配条件:
\begin{equation}
	\lim_{\xi \to \infty} y_1 = \lim_{x \to 0} y_0.
\end{equation}

\section{2}

sec. 9.2 4


\subsection{}

假设$y^{*}=a y$,$t^{*}=b t$,再引入参量$c$,由题意列方程组
$$m \frac{a}{b^{2}}=c \varepsilon,$$
$$\mu \frac{a}{b}=c,$$
$$k a=c,$$
$$\frac{m}{I} \frac{a}{b}=\varepsilon.$$

求解后可得,需要做的尺度化为
$$
y^{*}=\frac{I}{\mu} y, \quad t^{*}=\frac{\mu}{k} t, \quad \varepsilon=\frac{m k}{\mu^{2}}.
$$
\subsection{}
外部近似满足$y^{\prime}+y=0$,先算出外部近似解为
$$
y_{0}(t)=C \me^{-t}.
$$

引进$\xi \equiv t / \delta$,其中$\varepsilon \downarrow 0$时,$\delta(\varepsilon) \rightarrow 0$.再引进$y(\xi \delta, \varepsilon)=Y(\xi)$,可以得到
$$
\frac{\varepsilon}{\delta^{2}} \frac{d^{2} Y}{d \xi^{2}}+\frac{1}{\delta} \frac{d Y}{d \xi}+Y=0.
$$
在简化条件$\delta=\varepsilon$下,引进$y_{I}(\xi)=\lim _{\varepsilon \downarrow 0} Y(\xi)$,可以得到近似的内部方程
$$
\frac{d^{2} y_{I}}{d \xi^{2}}+\frac{d y_{I}}{d \xi}=0, \quad y_{I}(0)=0,\left.\quad \frac{d y_{I}}{d \xi}\right|_{\xi=0}=1.
$$
解得$y_{I}(\xi)=1-\me^{-\xi}$.

最后经过匹配可以确定外部近似函数的系数为$C=1$,即外部近似函数
$$
y_{0}(t)=\me^{-t}.
$$

\subsection{}
前面得到的两部分近似解都是比较符合实际情况的.本题的弹簧振子模型属于“强过阻尼”情况,首先总体曲线是类似于所得外部函数的指数衰减型,不会出现欠阻尼情况下的振动衰减,同时,初始时刻附近质量会有很大的速度,垂直偏离会有一个阶跃增长,该过程相对于后续过程是一个极快的过程,这些都是正如内部函数所展示的.

关于题干中已给尺度化的正确性,特别是在$t$不太大的情况下,首先估计$y^*$的范围,从初始时刻到达到最大垂直偏离$y_{\max }^{*}$,该过程持续的时间$\Delta t$非常小,因此该短暂过程的平均速度很大,内部阻尼起最主要作用,它产生的反冲量$\mu \bar{u} \Delta t=\mu\left(y_{\max }^{*} / \Delta t\right) \Delta t=\mu y_{\max }^{*}$近似于抵消掉$I$,进而估计出$y_{\max }^{*}=I / \mu$.再考虑时间尺度,前述过程时间很短,可以忽略,假设垂直偏离衰减到某个很小的设定值需要$t_{\max }^{*}$,该过程(从达到最大垂直偏离到衰减到某个很小的设定值)中只有内部阻尼和弹力作用,且两者强度整体基本相当,因为过程始末的动量基本都为0,可以表达为
$$
\mu \frac{y_{\max }^{*}}{t_{\max }^{*}} \sim k \frac{y_{\max }^{*}}{2}.
$$
由此可以估计时间尺度为$t_{\max }^{*} \sim \mu / k$.

\subsection{}
在上一小题最后的匹配过程中,已经求得了
$$
\lim _{\varepsilon \downarrow 0} y_{0}(\eta \Theta)=1.
$$
这里
$$
\lim _{\varepsilon \downarrow 0} \frac{\Theta}{\delta}=\infty, \quad \lim _{\varepsilon \downarrow 0} \Theta=0, \quad \eta=\frac{t}{\Theta}.
$$

因此在整个区域对$t>0$都成立的复合解为
$$
y_{U}(t)=y_{0}(t)+y_{I}\left(\frac{t}{\delta}\right)-\lim _{\varepsilon \downarrow 0} y_{0}(\eta \Theta)=\me^{-t}+\left(1-\me^{-\frac{t}{\varepsilon}}\right)-1=\me^{-t}-\me^{-\frac{t}{\varepsilon}}.
$$

\subsection{}
假设$m_{1}$和$m_{2}$是特征方程$\varepsilon m^{2}+m+1=0$的两个根,那么原方程在两个初始条件下的精确解为
$$
y(t)=\frac{\me^{m_{1} t}-\me^{m_{2} t}}{\left(m_{1}-m_{2}\right) \varepsilon}.
$$

对于方程$\varepsilon m^{2}+m+1=0$,在$0<\varepsilon \ll 1$的前提下,可以用级数方法得到一个近似解-1,再用初值迭代的逐次逼近方法得到另一个近似解$-\varepsilon^{-1}$,把这两个解代入到上面的结果中,可以得到$0<\varepsilon \ll 1$时的精确解为
$$
y(t)=\frac{\me^{-t}-\me^{-\frac{t}{\varepsilon}}}{(1-\varepsilon)} \approx \me^{-t}-\me^{-\frac{t}{\varepsilon}}.
$$
这与上一小题求得的复合解形式相同.



\nocite{*}

\bibliographystyle{plain}
\phantomsection

\addcontentsline{toc}{section}{参考文献} %向目录中添加条目,以章的名义
\bibliography{homework}

\end{document}
