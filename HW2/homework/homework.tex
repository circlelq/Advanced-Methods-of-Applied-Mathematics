\documentclass[12pt]{article}
\usepackage{natbib}
\usepackage{url}
\usepackage{amsmath}
\usepackage{graphicx}
\graphicspath{{fig/}}
\usepackage{parskip}
\usepackage{adjustbox}
\usepackage{fancyhdr}
\usepackage{commath}%定义d
\usepackage[UTF8,heading=true]{ctex}
\usepackage{bm}
\usepackage{titlesec}
\usepackage{caption}
\usepackage{paralist}
\usepackage{multirow}
\usepackage{booktabs} % To thicken table lines
\usepackage{titletoc}
\usepackage{diagbox}
\usepackage{bm}
\usepackage{autobreak}
\usepackage{authblk}
\usepackage{indentfirst}
\usepackage{float}
\usepackage{amsthm}
\usepackage{fontspec}
\usepackage{color}
%\usepackage{txfonts} %设置字体为times new roman
\usepackage{lettrine}
\usepackage{nameref}
%\usepackage[nottoc]{tocbibind}
\usepackage{amssymb}%font
\usepackage{lipsum}%make test words
\usepackage{picinpar}%words around the picture
\usepackage[all]{xy}%draw arrow
\usepackage{asymptote}%draw picture
\usepackage[perpage]{footmisc}%脚注每页清零
\usepackage[cmyk]{xcolor}

% \geometry{bottom=2.5cm,left=2cm,right=2cm,top=2.5cm}
\newcommand{\crefrangeconjunction}{ - }
\setlength{\parindent}{2em}

\ctexset{today=big}%日期类型设置


% ======================================
% = Color de la Universidad de Sevilla =
% ======================================
\usepackage{tikz}
\definecolor{PKUred}{cmyk}{0,1,1,0.45}
%超链接设置
\usepackage[breaklinks,colorlinks,linkcolor=PKUred,citecolor=PKUred,pagebackref,urlcolor=black]{hyperref}
\usepackage{cleveref}


\renewcommand*\footnoterule{%
    \vspace*{-3pt}%
    {\color{PKUred}\hrule width 2in height 0.4pt}%
    \vspace*{2.6pt}%
}





% %%% Equation and float numbering
% \numberwithin{equation}{section}		% Equationnumbering: section.eq#
% \numberwithin{figure}{section}			% Figurenumbering: section.fig#
% \numberwithin{table}{section}				% Tablenumbering: section.tab#


%代码设置
\usepackage{listings}
\usepackage{fontspec} % 定制字体
\newfontfamily\menlo{Menlo}
\usepackage{xcolor} % 定制颜色
\definecolor{mygreen}{rgb}{0,0.6,0}
\definecolor{mygray}{rgb}{0.5,0.5,0.5}
\definecolor{mymauve}{rgb}{0.58,0,0.82}
\lstset{ %
backgroundcolor=\color{white},      % choose the background color
basicstyle=\footnotesize\ttfamily,  % size of fonts used for the code
columns=fullflexible,
tabsize=4,
breaklines=true,               % automatic line breaking only at whitespace
captionpos=b,                  % sets the caption-position to bottom
commentstyle=\color{mygreen},  % comment style
escapeinside={\%*}{*)},        % if you want to add LaTeX within your code
keywordstyle=\color{blue},     % keyword style
stringstyle=\color{mymauve}\ttfamily,  % string literal style
frame=single,
rulesepcolor=\color{red!20!green!20!blue!20},
% identifierstyle=\color{red},
language=c++,
xleftmargin=4em,xrightmargin=2em, aboveskip=1em,
framexleftmargin=2em,
numbers=left
}

%脚注
\renewcommand\thefootnote{\fnsymbol{footnote}}

%定义常数i、e、积分符号d
\newcommand\mi{\mathrm{i}}
\newcommand\me{\mathrm{e}}

%%% Maketitle metadata
\newcommand{\horrule}[1]{\rule{\linewidth}{#1}} 	% Horizontal rule
\newcommand{\tabincell}[2]{\begin{tabular}{@{}#1@{}}#2\end{tabular}}

\newcommand{\chuhao}{\fontsize{42pt}{\baselineskip}\selectfont}
\newcommand{\xiaochuhao}{\fontsize{36pt}{\baselineskip}\selectfont}
\newcommand{\yihao}{\fontsize{28pt}{\baselineskip}\selectfont}
\newcommand{\erhao}{\fontsize{21pt}{\baselineskip}\selectfont}
\newcommand{\xiaoerhao}{\fontsize{18pt}{\baselineskip}\selectfont}
\newcommand{\sanhao}{\fontsize{15.75pt}{\baselineskip}\selectfont}
\newcommand{\sihao}{\fontsize{14pt}{\baselineskip}\selectfont}
\newcommand{\xiaosihao}{\fontsize{12pt}{\baselineskip}\selectfont}
\newcommand{\wuhao}{\fontsize{10.5pt}{\baselineskip}\selectfont}
\newcommand{\xiaowuhao}{\fontsize{9pt}{\baselineskip}\selectfont}
\newcommand{\liuhao}{\fontsize{7.875pt}{\baselineskip}\selectfont}
\newcommand{\qihao}{\fontsize{5.25pt}{\baselineskip}\selectfont}
\setcounter{secnumdepth}{2}
\usepackage{bm}
\usepackage{autobreak}
\usepackage{amsmath}
\setlength{\parindent}{2em}


%pdf文件设置
\hypersetup{
	pdfauthor={袁磊祺},
	pdftitle={高等应用数学作业2}
}

\title{
		\vspace{-1in} 	
		\usefont{OT1}{bch}{b}{n}
		\normalfont \normalsize \textsc{\LARGE Peking University}\\[1cm] % Name of your university/college \\ [25pt]
		\horrule{0.5pt} \\[0.5cm]
		\huge \bfseries{高等应用数学作业2} \\
		\horrule{2pt} \\[0.5cm]
}
\author{
		\normalfont 								\normalsize
		第二组\quad 袁磊祺 \quad 刘志如 \quad 宋庭鉴 \quad 岐亦铭 \quad 董淏翔 \quad 周子铭 \quad 撒普尔\\	\normalsize
        \today
}
\date{}

\begin{document}

%%%%%%%%%%%%%%%%%%%%%%%%%%%%%%%%%%%%%%%%%%%%%%
\renewcommand\contentsname{\vspace*{-2cm}\centerline{\textsf{目\quad 录}}\vspace*{-1.5cm}}
\renewcommand\listfigurename{插图目录}
\renewcommand\listtablename{表格目录}
\renewcommand\refname{参考文献}
\renewcommand\indexname{索引}
\renewcommand\figurename{图}
\renewcommand\tablename{表}
\renewcommand\abstractname{摘\quad 要}
\renewcommand\partname{部分}
\renewcommand\appendixname{附录}
\def\equationautorefname{式}%
\def\footnoteautorefname{脚注}%
\def\itemautorefname{项}%
\def\figureautorefname{图}%
\def\tableautorefname{表}%
\def\partautorefname{篇}%
\def\appendixautorefname{附录}%
\def\chapterautorefname{章}%
\def\sectionautorefname{节}%
\def\subsectionautorefname{小小节}%
\def\subsubsectionautorefname{subsubsection}%
\def\paragraphautorefname{段落}%
\def\subparagraphautorefname{子段落}%
\def\FancyVerbLineautorefname{行}%
\def\theoremautorefname{定理}%
\crefname{figure}{图}{图}
\crefname{equation}{式}{式}
\crefname{table}{表}{表}




\maketitle

\section{1}

\begin{equation}
	33^{1 / 5}=(32+1)^{1 / 5}=2\left(1+\frac{1}{32}\right)^{1 / 5}=2\left(1+\frac{1}{5} \times \frac{1}{32}-\frac{R}{2}\right)=\frac{161}{80}-R
\end{equation}

\begin{equation}
	\mathrm{R}=\frac{1}{5} \times \frac{4}{5} \times \frac{1}{32^{2}}+O\left(\frac{1}{32^{3}}\right) \approx \frac{1}{25 \times 256} \approx \frac{1}{6 \times 1000}=\frac{10}{6} \times 10^{-4} \approx 1.7 \times 10^{-4}
\end{equation}

\section{2}

\begin{equation}
	36x^3 + (162 + 4\epsilon)x^2-24\epsilon x - 9\epsilon = 0.
\end{equation}

\begin{equation}
	36x^2\left(x + \frac{9}{2}\right) = - \epsilon (4x^2 - 24x + 9)
\end{equation}

\begin{enumerate}
	\item $x\not= 0$,
	\begin{equation}
		x + \frac{9}{2} = - \epsilon\left(\frac{1}{9}-\frac{2}{3x}+\frac{1}{4x^2}\right)
	\end{equation}
	$x_0 = -\frac{9}{2}$,
	\begin{equation}
		x_1 = -\epsilon\left(\frac{1}{9}-\frac{2}{3x_0}+\frac{1}{4x_0^2}\right) - \frac{9}{2} = - \frac{9}{2} - \frac{22}{81} \epsilon
	\end{equation}
	\item $x_0 = 0$,
	\begin{equation}
		x^2 = -\epsilon \frac{4x_0^2-24x_0+9}{36\left(x_0+\frac{9}{2}\right) }
	\end{equation}
	\begin{equation}
		x_{2,3} = \pm \mi \frac{\sqrt{2\epsilon}}{6}
	\end{equation}
\end{enumerate}
以上为一阶近似的3个解。

\section{3}

原式可化为
\begin{equation}
	\left(x-\frac{1}{2}\right)^2 = \epsilon \frac{x}{x+4},\ \epsilon = \frac{1}{8}
\end{equation}
$x_0 = \frac{1}{2}$
\begin{equation}
	\left(x_1-\frac{1}{2}\right)^2 = \frac{\epsilon}{9}
\end{equation}
\begin{equation}
	x_{1,2} = \frac{1}{2} \pm \frac{\sqrt{2}}{12}
\end{equation}







\nocite{*}

\bibliographystyle{plain}
\phantomsection

\addcontentsline{toc}{section}{参考文献} %向目录中添加条目,以章的名义
\bibliography{homework}

\end{document}
