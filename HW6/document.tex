\documentclass[UTF8]{ctexart}

\CTEXsetup[format={\Large\bfseries}]{section}
\usepackage{graphicx}

\begin{document}
	
	\section{第6题}
	这是一个可以求解的方程,根据初始条件我们可以得到解为:
	\[f=\left ( 1-\frac{\varepsilon }{\pi ^{2}}\right )cos\left ( \pi x\right )+\frac{\varepsilon }{\pi ^{2}}+\left ( \frac{-1}{\pi ^{3}-\pi }\right )sin\left ( \pi x\right )+\frac{1}{\pi ^{2}-1}sinx\]
	取$\varepsilon=0$与$\varepsilon=0.1$对比得到图像:
	
	\includegraphics[width = 1.3\textwidth]{0.1.jpg}
	
	在原式中$\varepsilon=$在大部分时候绝对值都小于$sinx$,正如书上所说,无害的非正统函数的一种情形,也就是振荡项具有相对大的绝对值,只在狭小区域内会被超越,这种具有非正统的小区间在这个例子中可以忽略。根据图像也可以看到只在各个极小值处能表现出一些分离的情况

	
\end{document}