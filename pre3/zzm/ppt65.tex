\documentclass[12pt]{beamer}

\usepackage[UTF8]{ctex}
\usepackage{bm}
\usepackage{cleveref}
\usepackage{hyperref}
\usetheme[progressbar=frametitle]{metropolis}



\begin{document}

\begin{frame}[allowframebreaks]{6.5.1线性矩阵不等式}
线性矩阵不等式具有一般的描述形式
\\$\boldsymbol{F}\left ( x\right )=\boldsymbol{F}_{0}+x_{1}\boldsymbol{F}_{1}+\cdots +x_{m}\boldsymbol{F}_{m}< 0$
\\其中:$\boldsymbol{x}=\left [ x_{1},x_{2},\cdots x_{m},\right ]^{T}$ ,为多项式向量,$\boldsymbol{F}_{1}$为实对称矩阵或者复Hermite矩阵。

矩阵小于零表示矩阵负定。有$\boldsymbol{x}$解集为凸集,即:
\\$ \boldsymbol{F}\left [ \alpha \boldsymbol{x}_{1}+\left ( 1-\alpha \right )\boldsymbol{x}_{2}\right ]=\alpha \boldsymbol{F}\left ( \boldsymbol{x}_{1} \right ) +\left ( 1-\alpha \right )\boldsymbol{F}\left ( \boldsymbol{x}_{2} \right )< 0$
\\其中$0< \alpha < 1$,该解称为可行解。


\end{frame}

\begin{frame}[allowframebreaks]{6.5.1线性矩阵不等式}
若有$\boldsymbol{F}_{1}\left ( x\right )< 0$与$\boldsymbol{F}_{2}\left ( x\right )< 0$则

\[\begin{bmatrix}
\boldsymbol{F}_{1}\left ( x\right ) & 0\\ 
0 & \boldsymbol{F}_{2}\left ( x\right )
\end{bmatrix}< 0\]
同理有系列$\boldsymbol{F}_{i}\left ( x\right )< 0$, $i=1,2,...,k$
可有\[\boldsymbol{F}\left ( x\right )=\begin{bmatrix}
\boldsymbol{F}_{1}\left ( x\right ) &  &  & \\ 
&  \boldsymbol{F}_{2}\left ( x\right ) & & \\ 
&  & \cdots & \\ 
&  &  & \boldsymbol{F}_{k}\left ( x\right )
\end{bmatrix}< 0\]
\end{frame}

\begin{frame}[allowframebreaks]{6.5.2Lyapunov不等式}
若给定正定矩阵$\boldsymbol{Q}$,有Lyapunov方程:
\\$\boldsymbol{A}^{T}\boldsymbol{X}+\boldsymbol{X}\boldsymbol{A}=-\boldsymbol{Q}$
\\若存在正定的解$\boldsymbol{X}$则该系统稳定。上述问题很自然地表示成对Lyapunov不等式的求解\[\boldsymbol{A}^{T}\boldsymbol{X}+\boldsymbol{X}\boldsymbol{A}< 0\]
$\boldsymbol{X}$是对称矩阵,可以用一个$n\left ( n+1\right )/2$哥元素组成的向量$\boldsymbol{x}$描述:
\[x_{i-j+1+\left ( 2n-j+2\right )\left ( j-1\right )/2}=X_{i,j}\]
其中$j=1,..,n$而$i=j,...,n$。
\end{frame}

\begin{frame}[allowframebreaks]{6.5.2Lyapunov不等式}
接着便可以将\[\boldsymbol{A}^{T}\boldsymbol{X}+\boldsymbol{X}\boldsymbol{A}< 0\]
转化为\[\begin{bmatrix}
\boldsymbol{F}_{1} &  &  & \\ 
& \boldsymbol{F}_{2} &  & \\ 
&  & \cdots  & \\ 
&  &  & \boldsymbol{F}_{\frac{n\left ( n+1\right )}{2}}
\end{bmatrix}\cdot \boldsymbol{x}=\boldsymbol{F}\left ( \boldsymbol{x}\right )< 0 \]
matlab中用F
\end{frame}

\begin{frame}[allowframebreaks]{6.5.2一般代数Riccati不等式}
我们知道若\[\boldsymbol{F}\left ( \boldsymbol{x}\right )=\begin{bmatrix}
\boldsymbol{F}_{11}\left ( \boldsymbol{x}\right ) & \boldsymbol{F}_{12}\left ( \boldsymbol{x}\right )\\ 
\boldsymbol{F}_{21}\left ( \boldsymbol{x}\right ) & \boldsymbol{F}_{22}\left ( \boldsymbol{x}\right )
\end{bmatrix}\]是负定的,则有一下三个等价表达:
\[\boldsymbol{F}\left ( \boldsymbol{x}\right )< 0\]\[\boldsymbol{F}_{11}\left ( \boldsymbol{x}\right )< 0,\boldsymbol{F}_{22}\left ( \boldsymbol{x}\right )-\boldsymbol{F}_{21}\left ( \boldsymbol{x}\right )\boldsymbol{F}_{11}^{-1}\left ( \boldsymbol{x}\right )\boldsymbol{F}_{12}\left ( \boldsymbol{x}\right )< 0\]\[\boldsymbol{F}_{22}\left ( \boldsymbol{x}\right )< 0,\boldsymbol{F}_{11}\left ( \boldsymbol{x}\right )-\boldsymbol{F}_{12}\left ( \boldsymbol{x}\right )\boldsymbol{F}_{22}^{-1}\left ( \boldsymbol{x}\right )\boldsymbol{F}_{21}\left ( \boldsymbol{x}\right )< 0\]
\end{frame}

\begin{frame}[allowframebreaks]{6.5.2一般代数Riccati不等式}
对于更复杂的Riccati方程变换可以得到Riccati不等式
\[\boldsymbol{A}^{T}\boldsymbol{X}+\boldsymbol{X}\boldsymbol{A}+\left ( \boldsymbol{X}\boldsymbol{B}-\boldsymbol{C}^{T}\right )\boldsymbol{R}^{-1}\left ( \boldsymbol{X}\boldsymbol{B}-\boldsymbol{C}^{T}\right )^{T}< 0\]
对比
\[\boldsymbol{F}_{22}\left ( \boldsymbol{x}\right )< 0,\boldsymbol{F}_{11}\left ( \boldsymbol{x}\right )-\boldsymbol{F}_{12}\left ( \boldsymbol{x}\right )\boldsymbol{F}_{22}^{-1}\left ( \boldsymbol{x}\right )\boldsymbol{F}_{21}\left ( \boldsymbol{x}\right )< 0\]
可等价构建为
\[\boldsymbol{X}> 0,\begin{bmatrix}
\boldsymbol{A}^{T}\boldsymbol{X}+\boldsymbol{X}\boldsymbol{A} & \boldsymbol{XB}-\boldsymbol{C}^{T}\\ 
\boldsymbol{B}^{T}\boldsymbol{X}-\boldsymbol{C}& -\boldsymbol{R}
\end{bmatrix}< 0\]
\end{frame}
\begin{frame}[allowframebreaks]{6.5.3线性矩阵不等式问题分类}
(1)可行性问题,即:
\[\boldsymbol{F}\left ( \boldsymbol{x}\right )< 0\]
存在解

(2)线性目标函数最优化问题,即求:
\\	\qquad \qquad\qquad\qquad\qquad\qquad min  
$
\boldsymbol{c}^{T}\boldsymbol{x}
$
\\\qquad\qquad\qquad\qquad\qquad
$
\boldsymbol{x}\;s.t.
\boldsymbol{F}\left ( \boldsymbol{x}\right )< 0
$

(3)广义特征值最优化问题,即求:

	\qquad \qquad\qquad\qquad\qquad\qquad min 
$
\lambda 
$
\\\qquad\qquad\qquad\qquad
$
\lambda $ $\textbf{x}\;s.t.
\left\{
\begin{array}{ccc}
\boldsymbol{A}\left ( \boldsymbol{x}\right )< \lambda \boldsymbol{B}\left ( \boldsymbol{x}\right ) \\
\boldsymbol{B}\left ( \boldsymbol{x}\right )> 0\\
\boldsymbol{C}\left ( \boldsymbol{x}\right )< 0 
\end{array} 
\right.
$

\end{frame}

\begin{frame}[allowframebreaks]{6.5.4线性矩阵问题matlab求解}

考虑Riccati不等式
\[\boldsymbol{A}^{T}\boldsymbol{X}+\boldsymbol{X}\boldsymbol{A}+\left ( \boldsymbol{X}\boldsymbol{B}\right )\boldsymbol{R}^{-1} \boldsymbol{B}^{T}\boldsymbol{X}+\boldsymbol{Q}< 0\]
分块为\[\begin{bmatrix}
\boldsymbol{A}^{T}\boldsymbol{X}+\boldsymbol{X}\boldsymbol{A}+\boldsymbol{Q} &\boldsymbol{X}\boldsymbol{B} \\ 
\boldsymbol{B}^{T}\boldsymbol{X} & -\boldsymbol{R}
\end{bmatrix}< 0\]

\end{frame}

\begin{frame}[allowframebreaks]{6.5.4线性矩阵问题matlab求解}

带入:$\boldsymbol{A}=\begin{bmatrix}
-2 & -3 & -1\\ 
-3 & -1 & -1\\ 
1 & 0 & -4
\end{bmatrix}$

$\boldsymbol{B}=\begin{bmatrix}
-1 & 0 \\ 
0 & -1\\ 
-1 & -1
\end{bmatrix}$
$\boldsymbol{Q}=\begin{bmatrix}
-2 & 1 & -2\\ 
1 & -2 & -4\\ 
-2 & -4 & -2
\end{bmatrix}$
$\boldsymbol{R}=\begin{bmatrix}
1 & 0 \\ 
0 & 1\\ 
\end{bmatrix}$

得到\[\boldsymbol{X}=\begin{bmatrix}
0.9101 & 0.4641 & -0.2354\\ 
0.4641 & 0.7893 & -0.0383\\ 
-0.2354 & -0.0383 & 1.3616
\end{bmatrix}\]


\end{frame}

\begin{frame}[allowframebreaks]{6.5.5基于Yalmip工具箱最优化求解方法}
YALMIP工具箱解决线性规划;
\\	\qquad \qquad\qquad\qquad min \qquad 
$
6x_{1}-3x_{2}-5x_{3}-2x_{4}-9x_{5} 
$
\\\qquad\qquad\qquad
  $\textbf{x}\;s.t.
\left\{
\begin{array}{ccc}
2x_{2}+x_{3}+4x_{4}+2x_{5} \leq 54 \\
3x_{1}+4x_{2}+5x_{3}-x_{4}-x_{5} \leq 66\\
x_{1}\geq 0,x_{2}\geq 2,x_{3}\geq3, x_{4}\geq 0.5,x_{5} \geq 2
\end{array} 
\right.
$
得到$x_{1}=22,x_{2}= 2,x_{3}=3, x_{4}= 0.5,x_{5} = 22.5$
\\$
6x_{1}-3x_{2}-5x_{3}-2x_{4}-9x_{5} =-356.5
$

\end{frame}

\end{document}