\documentclass[12pt]{beamer}

\usepackage[UTF8]{ctex}
\usepackage{bm}
\usepackage{cleveref}
\usepackage{hyperref}
\usetheme[progressbar=frametitle]{metropolis}



\begin{document}

\begin{frame}[allowframebreaks]{6.6多目标优化模型}
例6-58  设商店有糖果A1,A2,A3,价格分别为4,2.8,2.4元/kg。要求购买不超过20元,总重不超过6kg。A1+A2总重不少于3kg,如何购买?
\\应该设立两个目标函数,一个是花钱最少,一个是重量最总,其他条件可以认为是约束条件。假设购买三种糖果的重量分别为x1, x2, x3kg,这时目标函数分别为
\\花钱:$f_1(x) = 4x_1+2.8x_2+2.4x_3 \rightarrow min$
\\重量:$f_2(x) = x_1+x_2+x_3 \rightarrow max$ 
那么模型该如何设立呢?
\end{frame}
\begin{frame}[allowframebreaks]{多目标优化模型}
	\qquad min\qquad \qquad \qquad \qquad \qquad
	$
	\left[
	\begin{array}{ccc}
	4x_1+2.8x_2+2.4x_3	\\
	x_1+x_2+x_3	
	\end{array} 
	\right]
	$
	\\
	$
	\textbf{x}\;s.t.
	\left\{
	\begin{array}{ccc}
	4x_1+2.8x_2+2.4x_3\leq20\\
	x_1+x_2+x_3\geq6\\
	x_1+x_2\geq3 \\
	x_1,x_2,x_3\geq0 
	\end{array} 
	\right.
	$
	\\从这之中我们可以得出多目标优化模型的一般表示形式:
	$$ \qquad \ J\qquad = \qquad min\qquad \textbf{F(x)}$$
	$$ \textbf{x} s.t.  \textbf{G(x)}\leq0$$
	其中,$\textbf{F(x)} = [f_1(\textbf{x}),f_2(\textbf{x}),f_3(\textbf{x}),\cdot \cdot \cdot,f_p(\textbf{x})]^T$
\end{frame}
\begin{frame}[allowframebreaks]{多目标问题转化为单目标问题求解}
	那么如何解这类问题呢?
	\\下面介绍三种转换方法:
	
	\\ \qquad (1)线性加权变换及求解
	\\ \qquad (2)线性规划问题的最佳妥协解
	\\ \qquad (3)线性规划问题的最小二乘解
\end{frame}
\begin{frame}[allowframebreaks]{(1)线性加权变换及求解}
	根据两个指标的侧重情况引入加权,目标函数改写为:
	$$f(\textbf{x}) = w_1f_1(\textbf{x})+ w_2f_2(\textbf{x})+ w_3f_3(\textbf{x})+\cdot \cdot \cdot+w_pf_p(\textbf{x})$$
	其中,$w_1+w_2+\cdot \cdot \cdot+w_p=1$, $0\leq w_1,w_2,\cdot \cdot \cdot,w_p\leq 1$.	
	\\则例6-58就可以改写成下面的线性规划系数\\
	\qquad min\qquad \qquad \qquad \qquad \qquad
	$(w_1[4,2.8,2.4]-w_2[1,1,1])\textbf{x}$
	\\
	$
	\textbf{x}\;s.t.
	\left\{
	\begin{array}{ccc}
		4x_1+2.8x_2+2.4x_3\leq20\\
		x_1+x_2+x_3\geq6\\
		x_1+x_2\geq3 \\
		x_1,x_2,x_3\geq0 
	\end{array} 
	\right.
	$
	\\(演示将在Matlab上进行)
\end{frame}

\begin{frame}[allowframebreaks]{(2)线性规划问题的最佳妥协解}
	考虑一类特殊的线性规划问题\\
	$$\quad \textbf{J} = \qquad max \qquad \textbf{Cx}$$
		$$
	\textbf{x}\;s.t.
	\left\{
	\begin{array}{ccc}
		Ax\leq B\\
		A_{eq}x=B_{eq}\\
		x_m\leq x\leq x_M
	\end{array} 
	\right.
	$$
	%此处的A,B,x应该加粗但是不知道咋回事,加粗不了
	目标函数的\textbf{C}不是一个向量,而是一个矩阵。
	\\每一个目标函数$j_i(x) = c_ix,i=1,2,\cdot \cdot \cdot , p$,可以理解为第i方的利益分配,所以这样的最优化问题可以认为是各方利益的最大分配。
	\\最佳妥协解的求解步骤如下:
	\\1. 单独求解每个单目标函数的最优化问题,得出最优解$f_k, k=1,2,\cdot \cdot \cdot , p$
	\\2. 通过规范化构造单独的目标函数
	$$f(x) = -\frac{1}{f_1}c_1x-\frac{1}{f_2}c_2x-\cdot \cdot \cdot -\frac{1}{f_p}c_px$$
	3. 最佳妥协解可以变换成下面的单目标限性规划问题并直接求解
	$$\quad \textbf{J} = \qquad max \qquad \textbf{f(x)}$$
	$$
	\textbf{x}\;s.t.
	\left\{
	\begin{array}{ccc}
		Ax\leq B\\
		A_{eq}x=B_{eq}\\
		x_m\leq x\leq x_M
	\end{array} 
	\right.
	$$
\end{frame}

\begin{frame}[allowframebreaks]{(3)线性规划问题的最小二乘解}
	考虑下面多目标线性规划问题的最小二乘表示
		$$\quad \qquad max \qquad \frac{1}{2}||Cx-d||^2$$
	$$
	\textbf{x}\;s.t.
	\left\{
	\begin{array}{ccc}
		Ax\leq B\\
		A_{eq}x=B_{eq}\\
		x_m\leq x\leq x_M
	\end{array} 
	\right.
	$$
	则最小二乘解可以由$x = lsqlin(C,d,A,B,A_{eq},B_{eq},x_m,x_M,x_0,options)$函数直接得到。

\end{frame}

\end{document}